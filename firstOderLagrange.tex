\documentclass{article}
\usepackage[margin=0.9in]{geometry}
\usepackage{graphicx} % Required for inserting images
\usepackage{amssymb}
\usepackage{amsmath}
\usepackage{amsthm}

\title{Introduction into Finite Elements and Algorithms}
\author{Group 3}
\date{November 2023}


\begin{document}

\maketitle

\section{Weak form of the problem}
\noindent Strong form of the problem:
\[
\begin{array}{rcl}
     -u''(x) = f(x)& \forall x\in [0,1] \\
     u(0)=0& \\
     u'(1)=0& \\     
\end{array}
\]

\noindent If we suppose that $u$ is a solution for the strong form, given any function $v$, it is also true that:
\begin{equation*}
    -u''(x)v(x) = f(x)v(x) \quad \forall x\in [0,1]
\end{equation*}

\noindent If we integrate that expression in [0,1], we arrive at:
\begin{equation*}
    -\int_0^1 u''(x)v(x)dx = \int_0^1 f(x)v(x) dx
\end{equation*}

\noindent For that integrals to be well defined, we assume that $u\in H^2([0,1])$, $v\in H^1([0,1])$ and $f\in L^2([0,1])$. Then, we integrate by parts:
\begin{equation*}
    -[u'(x)v(x)]_{x=0}^{1}-\int_0^1 u'(x)v(x)'=dx \int_0^1 f(x)v(x) dx \quad\forall v\in H^1([0,1])
\end{equation*}
\begin{equation*}
    -u'(1)v(1)+u'(0)v(0)+\int_0^1 u'(x)v(x)'=dx \int_0^1 f(x)v(x) dx \quad\forall v\in H^1([0,1])
\end{equation*}

\noindent To force a solution that verifies the Dirichlet bound conditions, we define:
\begin{equation*}
    V=\{ v\in H^1([0,1]) \, : \, v(0)=0\}
\end{equation*}

\noindent Using that functional space and given that u'(1)=0, we then arrive at the weak form of the problem:
\begin{equation}
    \int_0^1 u'(x)v(x)'=dx \int_0^1 f(x)v(x) dx \quad\forall v\in V
\end{equation}

\section{FEM with first order Lagrange elements}
\noindent Solve:
\begin{equation*}
    Ac=b
\end{equation*}

\noindent Where:
\begin{equation*}
    A=[a_{ij}]
\end{equation*}

\begin{gather}
     a_{ij}=\int_a^b \psi_j' \psi_i' dx \;\;\;\;\text{if } j\in\{ i-1,i,i+1\}  \\
     a_{ij}=0 \;\;\;\;\text{if } j\notin\{ i-1,i,i+1\} \\
     b=[b_i], \quad b_{i}=\int_a^b f \psi_i dx \quad
     i=1,...,N+1 \\
     b_1 = b_{N+1} = 0 
\end{gather}


\noindent The functions $\psi_i$ are defined as:
\[\psi_i= \left\{
\begin{array}{rcl}
\frac{x-x_{i-1}}{h}& \text{if } x\in [x_{i-1}, x_i]\\
\frac{x_{i+1}-x}{h}& \text{if } x\in [x_{i}, x_{i+1}]\\
0& \text{if } x\notin [x_{i-1}, x_{i+1}]
\end{array}\right.\]


\end{document}
