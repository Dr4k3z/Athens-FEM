\documentclass{article}
\usepackage[margin=0.9in]{geometry}
\usepackage{graphicx} % Required for inserting images
\usepackage{mathdots}
\usepackage{amssymb}
\usepackage{amsmath}
\usepackage{amsthm}

\title{Introduction into Finite Elements and Algorithms}
\author{Group 3}
\date{November 2023}


\begin{document}

\maketitle

\noindent Given the bi-harmonic equation for the 1D problem:

\[ (P) \left\{
\begin{array}{rcl}
u^{4)}=f \quad \text{in } [0,1]\\
u(0)=0\\
u'(0)=0\\
u''(1)=0\\
u'''(1)=0
\end{array}
\right.\]

\noindent We approach the problem by defining the function $w=u''$, which lead us to the breakdown in the following problems:

\[ (P1) \left\{
\begin{array}{rcl}
w''=f \quad \text{in } [0,1]\\
w(1)=0\\
w'(1)=0
\end{array}
\right.\]

\[ (P2) \left\{
\begin{array}{rcl}
u''=w \quad \text{in } [0,1]\\
u(0)=0\\
u'(0)=0
\end{array}
\right.\]

\paragraph{Weak form for (P1):}
\begin{gather*}
    w''=f \Rightarrow w''v=fv \quad \forall v\in V \Rightarrow \int_0^1 w''v \,dx=\int_0^1 fv \,dx\quad \forall v\in V
\end{gather*}

\noindent We now integrate the left side by parts:
\begin{gather*}
    [w'v]_{x=0}^{x=1}-\int_0^1 w'v'\, dx= \int_0^1 fv \,dx\quad \forall v\in V\\
    w'(1)v(1)-w'(0)v(0)-\int_0^1 w'v'\, dx= \int_0^1 fv \,dx\quad \forall v\in V\\
    w'(0)v(0)+\int_0^1 w'v'\, dx= \int_0^1 -fv \,dx\quad \forall v\in V\\
\end{gather*}

\paragraph{$1^{st}$ order Lagrange FEM for (P1):}
\noindent Let $x_1, \, x_2,...,x_{N+1}$ be the nodes of the one-dimensional mesh. According to the first order report, the vector b is written as:
\begin{gather*}
    b=\begin{pmatrix}
        b_1^1\\b_2^1+b_1^2\\b_2^2+b_1^3\\ \vdots \\ b_2^N
    \end{pmatrix}, \quad b^i_{l}=\int_{x_i}^{x_{i+1}} \varphi_{i-1+l} f dx \quad
     l=1; \quad 1\leq i \leq N
\end{gather*}

\noindent We approximate the result using the trapezoidal rule:
\begin{gather*}
    b^i_{l}=\int_{x_i}^{x_{i+1}} \varphi_{i-1+l} f dx=\frac{f(x_i)+f(x_{i+1})}{2h}
\end{gather*}

\noindent On the other hand, the stiffness matrix is given by the expression:
\begin{gather*}
    K=\begin{pmatrix}
        K_{11}^1 & K_{12}^1 & 0 & 0 & \cdots & 0\\
        K_{21}^1 & K_{22}^1+K_{11}^2 & K_{12}^2 & 0 & \cdots & 0\\ 
        0 & K_{21}^2 & K_{22}^2+K_{11}^3 & K_{12}^3 & \cdots & 0\\
        \vdots & \vdots & \vdots & \vdots & \ddots & \vdots\\
        0 & 0 & 0 & 0 & \cdots & K_{22}^N
    \end{pmatrix}
\end{gather*}

\noindent Where the coefficients $K_{lk}^1$ for $l,k=1,2$ and $1\leq i \leq N$ are defined as:
\begin{gather*}
    K^i_{lk}=\int_{x_i}^{x_{i+1}} \varphi_{i-1+l}' \varphi_{i-1+k}' dx= \int_{x_i}^{x_{i+1}} (-1)^l\frac{1}{h}(-1)^k\frac{1}{h}dx=\frac{(-1)^{l+k}}{h}
\end{gather*}

\noindent As a consequence,
\begin{gather*}
    K=\frac{1}{h}\begin{pmatrix}
        1 & -1 & 0 & 0 & \cdots & 0\\
        -1 & 1+1 & -1 & 0 & \cdots & 0\\ 
        0 & -1 & 1+1 & -1 & \cdots & 0\\
        \vdots & \vdots & \vdots & \vdots & \ddots & \vdots\\
        0 & 0 & 0 & 0 & \cdots & 1
    \end{pmatrix}=
    \frac{1}{h}\begin{pmatrix}
        1 & -1 & 0 & 0 & \cdots & 0\\
        -1 & 2 & -1 & 0 & \cdots & 0\\ 
        0 & -1 & 2 & -1 & \cdots & 0\\
        \vdots & \vdots & \vdots & \vdots & \ddots & \vdots\\
        0 & 0 & 0 & 0 & \cdots & 1
    \end{pmatrix}
\end{gather*}

\noindent There is no mass matrix in the problem, so we take A=K. 

% \noindent We shall express too the boundary conditions term in local coordinates:
% \begin{gather*}
%     w'(0)v(0)=\left( \begin{array}{cc}
%          v_1^1 & v_2^1
%     \end{array} \right)
%     \left( \begin{array}{cc}
%          {p_1^1}'(0) \\ {p_2^1}'(0)
%     \end{array} \right)
%     \left( \begin{array}{cc}
%          p_1^1(0) & p_2^1(0)
%     \end{array} \right)
%     \left( \begin{array}{cc}
%          w_1^1 \\ w_2^1
%     \end{array} \right)= \\
%     =\left( \begin{array}{cc}
%          v_1^1 & v_2^1
%     \end{array} \right)
%     \left( \begin{array}{cc}
%         {p_1^1}'(0) p_1^1(0) & {p_1^1}'(0) p_2^1(0) \\
%          p_1^1(0) {p_2^1}'(0) & {p_2^1}'(0) p_2^1(0)
%     \end{array} \right)
%     \left( \begin{array}{cc}
%          w_1^1 \\ w_2^1
%     \end{array} \right)=
%     \left( \begin{array}{cc}
%          v_1^1 & v_2^1
%     \end{array} \right)
%     \left( \begin{array}{cc}
%         -1 & 0 \\
%          1 & 0
%     \end{array} \right)
%     \left( \begin{array}{cc}
%          w_1^1 \\ w_2^1
%     \end{array} \right)
% \end{gather*}

% \noindent Finally, we model the boundary term in global coordinates with the following matrix:
% \begin{gather*}
%     Q=\begin{pmatrix}
%         -1 & 0 & 0 & \cdots & 0\\
%         1 & 0 & 0 & \cdots & 0\\
%         0 & 0 & 0 & \cdots & 0\\
%         \vdots & \vdots & \vdots & \ddots & \vdots\\
%         0 & 0 & 0 & \cdots & 0\\
%     \end{pmatrix}
% \end{gather*}

% \noindent With that information, we can conclude that the matrix $A$ needed for the linear system $Ac=w$ is the following one:
% \begin{gather*}
%     \frac{1}{h}\begin{pmatrix}
%         0 & -1 & 0 & 0 & \cdots & 0\\
%         0 & 2 & -1 & 0 & \cdots & 0\\ 
%         0 & -1 & 2 & -1 & \cdots & 0\\
%         \vdots & \vdots & \vdots & \vdots & \ddots & \vdots\\
%         0 & 0 & 0 & 0 & \cdots & 1
%     \end{pmatrix}
% \end{gather*}

% \begin{gather*}
%     K=[K_{ij}]_{ij}, \quad K_{ij}=\int_0^1 \psi_i'\psi_j'dx
% \end{gather*}

% \noindent Where the functions $\psi_i$ stand for the first order Lagrange elements shape functions in the element $E_i$. There is no mass matrix in this particular problem, although do we have the boundary conditions $w'(0)v(0)$. We also define:
% \begin{gather*}
%     b=[b_i]_{i}, \quad b_i=\int_0^1 -f\psi_i\, dx
% \end{gather*}

% \noindent We aim to express the FEM linear sistem in terms of local coordinates in each element. In this coordinates, the approximate solution can be written as:
% \begin{gather*}
%     {w^h}_{|_{Ei}}= w_1^i p_1^i+ w_2^i p_2^i = \left( \begin{array}{cc} p_1^i & p_2^i \end{array} \right) \left( \begin{array}{cc}
%          w_1^i \\ w_2^i
%     \end{array} \right)
% \end{gather*}

% \noindent Hence, we can rewrite de stiffness matrix in local coordinates for element $E_i$, $i\in\{1,..N\}$:
% \begin{gather*}
%     \int_{E_i} {w^h(x)}' {v^h(x)}'\, dx =\int_{E_i} \left( \begin{array}{cc}
%          v_1^i & v_2^i 
%     \end{array}  \right)
%     \left( \begin{array}{cc}
%          {p_1^i(x)}' {p_1^i(x)}' & {p_1^i(x)}' {p_2^i(x)}'  \\
%          {p_1^i(x)}' {p_2^i(x)}' & {p_2^i(x)}' {p_2^i(x)}'
%     \end{array} \right)
%     \left( \begin{array}{cc}
%          u_1^i \\
%          u_2^i
%     \end{array} \right) dx =\\
%     = \left( \begin{array}{cc}
%          v_1^i & v_2^i 
%     \end{array}  \right) \left[  
%     \int_{E_i} \left( \begin{array}{cc}
%          {p_1^i(x)}' {p_1^i(x)}' & {p_1^i(x)}' {p_2^i(x)}'  \\
%          {p_1^i(x)}' {p_2^i(x)}' & {p_2^i(x)}' {p_2^i(x)}'
%     \end{array} \right) dx
%     \right] \left( \begin{array}{cc}
%          u_1^i \\
%          u_2^i
%     \end{array} \right)
% \end{gather*}

% \begin{gather*}
%     K^i=[K^i_{ml}] \: m,l\in\{1,2\}, \quad K^i_{ml}=  \int_{E_i} {p_m^i}'(x) {p_l^i}'(x) \, dx
% \end{gather*}

% \noindent Proceeding in the same way, we arrive at the expression:
% \begin{gather*}
%     -\int_{E_i} f(x) v^h(x) \, dx = - \int_{E_i} f(x) \left( \begin{array}{cc}
%          v_1^i & v_2^i
%     \end{array} \right)
%     \left( \begin{array}{cc}
%          p_1^i(x) \\
%          p_2^i(x)
%     \end{array} \right) dx =
%     \left( \begin{array}{cc}
%          v_1^i & v_2^i
%     \end{array} \right) \int_{E_i} -f(x)
%     \left( \begin{array}{cc}
%          p_1^i(x) \\
%          p_2^i(x)
%     \end{array} \right) dx
% \end{gather*}

% \noindent As a consequence:
% \begin{gather*}
%     b^i=[b_k^i]_{k} \, k\in\{ 1,2 \}, \quad b_k^i=\int_{E_i} -f(x)p_k^i(x) \, dx
% \end{gather*}

% \noindent Making use of the previously introduced change of variables, $x=\frac{2}{h}y+x_i$ , we migrate the element to the interval [-1,1]. The vector $b^i$ can be written as follows:
% \begin{gather*}
%     b_k^i=\frac{2}{h} \int_{-1}^1 -f\big(\frac{2}{h}y+x_i\big)(-1)^k \frac{1}{2}y \, dy = (-1)^{k+1}h \int_{-1}^1 f\big(\frac{2}{h}y+x_i\big)y\, dy\\
%     b^i= \frac{1}{h} \int_{-1}^1 f\big(\frac{2}{h}x+x_i\big)x\, dx \left ( \begin{array}{cc}
%         1 \\ -1  
%     \end{array} \right)
% \end{gather*}



% \noindent We will now assembly the matrix of the FEM method. As we are using the first order Lagrange elements:
% \begin{gather*}
%     K=\begin{pmatrix}
%          -1 & 2 & \cdots & 0 & 0 & \cdots & 0 & 0\\
%          \cdots & \cdots & \ddots & \vdots & \vdots & \iddots & \cdots & \cdots \\
%          0 & 0 & \cdots & 2 & -1 & \cdots & 0 & 0\\
%          0 & 0 & \cdots & -1 & 2 & \cdots & 0 & 0\\
%          \cdots & \cdots & \iddots & \vdots & \vdots & \ddots & \cdots & \cdots \\
%          0 & 0 & \cdots & 0 & 0 & \cdots & 2 & -1\\
%          0 & 0 & \cdots & 0 & 0 & \cdots & -1 & 1
%     \end{pmatrix}
% \end{gather*}

% \begin{gather*}
%     b=\frac{1}{h}\begin{pmatrix}
%          \int_{-1}^1 f\big(\frac{2}{h}x+x_1\big)x\, dx\\
%          \int_{-1}^1 f\big(\frac{2}{h}x+x_2\big)x\, dx-\int_{-1}^1 f\big(\frac{2}{h}x+x_1\big)x\, dx\\
%          \int_{-1}^1 f\big(\frac{2}{h}x+x_3\big)x\, dx-\int_{-1}^1 f\big(\frac{2}{h}x+x_2\big)x\, dx\\
%          \vdots \\
%          \int_{-1}^1 f\big(\frac{2}{h}x+x_N\big)x\, dx- \int_{-1}^1 f\big(\frac{2}{h}x+x_{N-1}\big)x\, dx\\
%          -\int_{-1}^1 f\big(\frac{2}{h}x+x_N\big)x\, dx
%     \end{pmatrix}
% \end{gather*}


% \noindent We will now assembly the matrix of the FEM method. Due to the geometry of the problem with the first orden lagrangian elements, the change of matrix basis is:
% \begin{gather*}
%     \begin{pmatrix}
%          u_1^i \\ u_2^i
%     \end{pmatrix} =
%     \begin{pmatrix}
%          0 & 0 & \cdots & 1 & 0 & \cdots & 0 & 0\\
%          0 & 0 & \cdots & 0 & 1 & \cdots & 0 & 0
%     \end{pmatrix}
%     \begin{pmatrix}
%          u_1\\ u_2\\u_3\\ \vdots \\u_{N}\\u_{N+1} 
%     \end{pmatrix}
% \end{gather*}

% \noindent We define:
% \begin{gather*}
%     W^i=\begin{pmatrix}
%          0 & 0 & \cdots & 1 & 0 & \cdots & 0 & 0\\
%          0 & 0 & \cdots & 0 & 1 & \cdots & 0 & 0
%     \end{pmatrix}
% \end{gather*}

% \noindent Then, the stiffness matrix por the global coordinates is:
% \begin{gather*}
%     K=\sum\limits_{i=1}^{N} [W^i]^T K^i W^i= \sum\limits_{i=1}^{N}
%     \begin{pmatrix}
%          0 & 0\\ 0 & 0\\
%          \vdots & \vdots\\
%          0 & 1\\ 1 & 0\\
%          \vdots & \vdots\\
%          0 & 0\\ 0 & 0
%     \end{pmatrix}
%     \frac{1}{h}
%     \begin{pmatrix}
%         1&  -1\\
%      -1& 1
%     \end{pmatrix}
%     \begin{pmatrix}
%          0 & 0 & \cdots & 1 & 0 & \cdots & 0 & 0\\
%          0 & 0 & \cdots & 0 & 1 & \cdots & 0 & 0
%     \end{pmatrix} =\\
%     = \frac{1}{h}\sum\limits_{i=1}^{N} \begin{pmatrix}
%          0 & 0\\ 0 & 0\\
%          \vdots & \vdots\\
%          0 & 1\\ 1 & 0\\
%          \vdots & \vdots\\
%          0 & 0\\ 0 & 0
%     \end{pmatrix}
%     \begin{pmatrix}
%          0 & 0 & \cdots & 1 & -1 & \cdots & 0 & 0\\
%          0 & 0 & \cdots & -1 & 1 & \cdots & 0 & 0
%     \end{pmatrix} = 
%     \frac{1}{h}\sum\limits_{i=1}^{N}
%     \begin{pmatrix}
%          0 & 0 & \cdots & 0 & 0 & \cdots & 0 & 0\\
%          0 & 0 & \cdots & 0 & 0 & \cdots & 0 & 0\\
%          \cdots & \cdots & \ddots & \vdots & \vdots & \iddots & \cdots & \cdots \\
%          0 & 0 & \cdots & 1 & -1 & \cdots & 0 & 0\\
%          0 & 0 & \cdots & -1 & 1 & \cdots & 0 & 0\\
%          \cdots & \cdots & \iddots & \vdots & \vdots & \ddots & \cdots & \cdots \\
%          0 & 0 & \cdots & 0 & 0 & \cdots & 0 & 0\\
%          0 & 0 & \cdots & 0 & 0 & \cdots & 0 & 0
%     \end{pmatrix}=\\
%     \\=\frac{1}{h}\begin{pmatrix}
%          1 & -1 & \cdots & 0 & 0 & \cdots & 0 & 0\\
%          -1 & 2 & \cdots & 0 & 0 & \cdots & 0 & 0\\
%          \cdots & \cdots & \ddots & \vdots & \vdots & \iddots & \cdots & \cdots \\
%          0 & 0 & \cdots & 2 & -1 & \cdots & 0 & 0\\
%          0 & 0 & \cdots & -1 & 2 & \cdots & 0 & 0\\
%          \cdots & \cdots & \iddots & \vdots & \vdots & \ddots & \cdots & \cdots \\
%          0 & 0 & \cdots & 0 & 0 & \cdots & 2 & -1\\
%          0 & 0 & \cdots & 0 & 0 & \cdots & -1 & 1
%     \end{pmatrix}
% \end{gather*}

% \noindent Similarly, we can state:
% \begin{gather*}
%     b=\sum\limits_{i=1}^{N} [W^i]^T b^i W^i=\sum\limits_{i=1}^{N}
%     \begin{pmatrix}
%          0 & 0\\ 0 & 0\\
%          \vdots & \vdots\\
%          0 & 1\\ 1 & 0\\
%          \vdots & \vdots\\
%          0 & 0\\ 0 & 0
%     \end{pmatrix}
%     \frac{1}{h} \int_{-1}^1 f(x)x\, dx
%     \begin{pmatrix}
%         1 & -1
%     \end{pmatrix}
%     \begin{pmatrix}
%          0 & 0 & \cdots & 1 & 0 & \cdots & 0 & 0\\
%          0 & 0 & \cdots & 0 & 1 & \cdots & 0 & 0
%     \end{pmatrix}=\\
%     =\frac{1}{h} \int_{-1}^1 f(x)x\, dx\sum\limits_{i=1}^{N} 
%     \begin{pmatrix}
%          0 & 0\\ 0 & 0\\
%          \vdots & \vdots\\
%          0 & 1\\ 1 & 0\\
%          \vdots & \vdots\\
%          0 & 0\\ 0 & 0
%     \end{pmatrix}
%     \begin{pmatrix}
%          0 & 0 & \cdots & 1 & -1 & \cdots & 0 & 0\\
%     \end{pmatrix}
% \end{gather*}

\paragraph{Weak form for (P2):}
\noindent As stated before, we approach the problem:\\
\[ (P2) \left\{
\begin{array}{rcl}
u''=w \quad \text{in } [0,1]\\
u(0)=0\\
u'(0)=0
\end{array}
\right.\]

\begin{gather*}
    u''=w \Rightarrow u''v=w v \quad \forall v\in V \Rightarrow \int_0^1 u''v \,dx=\int_0^1 w v \,dx\quad \forall v\in V
\end{gather*}

\noindent We now integrate the left side by parts:
\begin{gather*}
    [u'v]_{x=0}^{x=1}-\int_0^1 u'v'\, dx= \int_0^1 w v \,dx\quad \forall v\in V\\
    u'(1)v(1)-u'(0)v(0)-\int_0^1 u'v'\, dx= \int_0^1 w v \,dx\quad \forall v\in V\\
    -u'(1)v(1)+\int_0^1 u'v'\, dx= \int_0^1 -w v \,dx\quad \forall v\in V\\
\end{gather*}

\paragraph{$1^{st}$ order Lagrange FEM for (P2):}
\noindent Analogous to the (P1) case, there is no mass matrix, and the problem matrix is the matrix A given in (P1).
% \[ \overline{K}^i=\frac{1}{h}\left( \begin{array}{cc}
%      1&  -1\\
%      -1& 1
% \end{array} \right) \]

% \noindent Taking $f=w$ in the previous case, we can also write:
% \[
% \overline{b}^i= \frac{1}{2} \int_{-1}^1 w(x)x\, dx \left ( \begin{array}{cc}
%         1\\ -1  
%     \end{array} \right)
% \]

% \noindent The last step is to properly write the boundary conditions. If the nodes are $\{ x_1,\, x_2,...,x_{N+1} \}$:
% \begin{gather*}
%     u'(1)v(1)=\left( \begin{array}{cc}
%          v_1^{N+1} & v_2^{N+1}
%     \end{array} \right)
%     \left( \begin{array}{cc}
%          {p_1^{N+1}}' \\ {p_2^{N+1}}'
%     \end{array} \right)
%     \left( \begin{array}{cc}
%          p_1^{N+1} & p_2^{N+1}
%     \end{array} \right)
%     \left( \begin{array}{cc}
%          u_1^{N+1} \\ u_2^{N+1}
%     \end{array} \right)= \\
%     =\left( \begin{array}{cc}
%          v_1^{N+1} & v_2^{N+1}
%     \end{array} \right)
%     \left( \begin{array}{cc}
%         {p_1^{N+1}}' p_1^{N+1} & {p_1^{N+1}}' p_2^{N+1} \\
%          p_1^{N+1} {p_2^{N+1}}' & {p_2^{N+1}}' p_2^{N+1}
%     \end{array} \right)
%     \left( \begin{array}{cc}
%          u_1^{N+1} \\ u_2^{N+1}
%     \end{array} \right)
% \end{gather*}

\paragraph{Coupling of (P1) and (P2):}

\noindent We reduced the approximation of problems (P1) and (P2) to solving the two linear systems:
\begin{gather*}
    Aw=b\\
    Au=w \Rightarrow Au-Iw=0
\end{gather*}

\noindent This is equivalent to the formulation:
\begin{gather*}
    \begin{pmatrix}
        A &-I\\
        0 & A
    \end{pmatrix}
    \begin{pmatrix}
        u \\ w
    \end{pmatrix}
    = \begin{pmatrix}
        0 \\ b
    \end{pmatrix}
\end{gather*}

\noindent We can then find $(u \: w)^T$ by solving the linear system, and then extract $u$ from the solution vector, finding the approximate solution of the problem (P).

\end{document}
