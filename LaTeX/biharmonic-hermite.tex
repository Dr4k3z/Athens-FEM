\documentclass{article}
\usepackage[margin=0.9in]{geometry}
\usepackage{graphicx} % Required for inserting images
\usepackage{mathdots}
\usepackage{amssymb}
\usepackage{amsmath}
\usepackage{amsthm}

\title{Introduction into Finite Elements and Algorithms}
\author{Group 3}
\date{November 2023}


\begin{document}
\maketitle

\noindent We aim to solve the bi-harmonic equation for the 1D problem:

\[ (P) \left\{
\begin{array}{rcl}
u^{4)}=f \quad \text{in } [0,1]\\
u(0)=0\\
u'(0)=0\\
u''(1)=0\\
u'''(1)=0
\end{array}
\right.\]

\paragraph{Weak form}
\begin{gather*}
    u^{4)}=f \Rightarrow u^{4)}v=fv \quad\forall v\in V \Rightarrow \int_0^1 u^{4)}v \,dx= \int_0^1 fv \,dx \quad\forall v\in V
\end{gather*}

\noindent We integrate by part the integral on the left side:
\begin{gather*}
    \int_0^1 fv \, dx=[u'''v]_{x=0}^1 -\int_0^1 u'''v'dx=u'''(1)v(1)-u'''(0)v(0) -\int_0^1 u'''v' \, dx \quad\forall v\in V
\end{gather*}

\noindent As u(0)=0, we impose $v\in V=\{w\in H^{2} \: : \: w(0)=0, \: w'(0)=0\}$. Then $u'''(0)v(0)=0$, and we also have enough regularity to integrate by parts again.
\begin{gather*}
    \int_0^1 fv \, dx = -[u''v']_{x=0}^1 + \int_0^1 u''v'' \, dx = -u''(1)v(1)+u''(0)v'(0) + \int_0^1 u''v'' \, dx=\int_0^1 u''v'' \, dx \quad \forall v\in V
\end{gather*}

\noindent Therefore, the weak problem is, given $f\in L^2([0,1])$, to find $u\in V$ such as for every $v\in V$ it is verified that:
\begin{equation}
    \int_0^1 u''v'' \, dx = \int_0^1 fv \, dx
\end{equation}

\paragraph{FEM with Hermite elements}\textbf{ }\\
\noindent We use shape functions $\psi_i \in \mathcal{C}^1$ such as for every element $E_i$ ${\psi_i}_{|_{E_i}}$ is a third degree polynomial. The local basis in the interval $[-1,1]$ is:
\begin{gather*}
         \psi_1= \frac{1}{4}[-x^3+3x+2]\\
         \psi_2= \frac{1}{4}[x^3-x^2-x+1]\\
         \psi_3= \frac{1}{4}[x^3-3x+2]\\
         \psi_4= \frac{1}{4}[x^3+x^2-x-1]\\
\end{gather*}

\noindent This shape functions have the following properties:
\begin{gather*}
    \begin{array}{cccc}
         \psi_1(-1)= 1; & {\psi_1}'(-1)=0; & \psi_1(1)= 0; & {\psi_1}'(-1)=0 \\
         \psi_2(-1)= 0; & {\psi_2}'(-1)=1; & \psi_2(1)= 0; & {\psi_2}'(-1)=0 \\
         \psi_3(-1)= 0; & {\psi_3}'(-1)=0; & \psi_3(1)= 1; & {\psi_3}'(-1)=0 \\
         \psi_4(-1)= 0; & {\psi_4}'(-1)=0; & \psi_4(1)= 0; & {\psi_4}'(-1)=1 \\
    \end{array}
\end{gather*}

\noindent The second derivative of this shape functions is:
\begin{gather*}
    \psi_j''(x)=\left\{ \begin{array}{cc}
         (-1)^{\frac{j+1}{2}}\frac{3}{2}x & \text{if } j\in\{1,3\}\\
         \frac{3}{2}x+2\cdot (-1)^{\frac{j}{2}} & \text{if } j\in\{2,4\}
    \end{array} \right.
\end{gather*}

\noindent We now discretize the weak form in elements, and change variables to make $[x_i,x_{i+1}]\rightarrow [-1,1]$. Then, we substitute the expression of the shape functions in order to obtain the coefficients of the FEM linear system:\\

\noindent OPTION 1: If $j,l\in{1,3}$:
\begin{gather*}
    \frac{2}{h}\int_{-1}^{1} {\psi_j}''(x){\psi_l}''(x) \, dx = \frac{2}{h}\int_{-1}^1 (-1)^{\frac{j+1}{2}}\frac{3}{2}x (-1)^{\frac{l+1}{2}}\frac{3}{2}x \, dx=(-1)^{\frac{j+l}{2}+1} \frac{2}{h}\frac{9}{4}\int_{-1}^{1} x^2dx=(-1)^{\frac{j+l}{2}+1} \frac{2}{h}\frac{9}{4}\Big[\frac{x^3}{3}\Big]_{x=-1}^1=\\
    = 3\cdot (-1)^{\frac{j+l}{2}+1}
\end{gather*}

\noindent OPTION 2: If $j,l\in{2,4}$:
\begin{gather*}
    \frac{2}{h}\int_{-1}^{1} {\psi_j}''(x){\psi_l}''(x) \, dx = \frac{2}{h}\int_{-1}^1 \big(\frac{3}{2}x+2\cdot (-1)^{\frac{j}{2}}\big)\big( \frac{3}{2}x+2\cdot (-1)^{\frac{l}{2}} \big)dx=\\
    =\frac{2}{h}\Big[\frac{9}{4}
    \int_{-1}^{1}x^2 dx+2\big( (-1)^{\frac{j}{2}}+(-1)^{\frac{l}{2}}\big)\int_{-1}^{1}x dx+4(-1)^{\frac{j+l}{2}}\int_{-1}^{1} dx\Big]= \frac{2}{h}\Big[ \frac{3}{2} +2 \big( (-1)^{\frac{j}{2}}+(-1)^{\frac{l}{2}}\big)+8(-1)^{\frac{j+l}{2}}\Big]
\end{gather*}

\noindent OPTION 3: If $j\in{1,3}$ and $l\in{2,4}$:
\begin{gather*}
    \frac{2}{h}\int_{-1}^{1} {\psi_j}''(x){\psi_l}''(x) \, dx = \frac{2}{h}\int_{-1}^1 
    (-1)^{\frac{j-1}{2}}\frac{3}{2}x \Big(\frac{3}{2}x+2\cdot (-1)^{\frac{l}{2}}\Big) dx= \frac{2}{h}\Big[ (-1)^{\frac{j-1}{2}}\frac{9}{4} \int_{-1}^{1}x^2dx+3\cdot (-1)^{\frac{j+l-1}{2}}\int_{-1}^{1} x   dx   \Big]=\\
    =\frac{2}{h}\Big[ (-1)^{\frac{j-1}{2}}\frac{3}{2}+3\cdot(-1)^{\frac{j+l-1}{2}}  \Big]
\end{gather*}

\noindent As a consequence, the elementary stiffness matrix is written as follows:
\begin{gather*}
    K^i=\frac{1}{h}\begin{pmatrix}
        -4/3 & -3  & 4/3 &  9 \\
         -3  &  11 &  3  & -13\\
         4/3 &  3  & -4/3&  -3\\
          9  & -13 &   -3&  27 
    \end{pmatrix}
\end{gather*}
\end{document}
